% !TEX root =  ICRA2012-Patil.tex

Explicitly considering uncertainty in robot motion and sensing during motion planning enables the computation of safer plans.
%When dealing with stochastic mobile systems, the uncertainty in the robot state that arises during execution precludes the use of deterministic motion planning algorithms \cite{Book:Lavalle06}.
Motion planners that consider uncertainty have been developed recently for a variety of domains and settings \cite{Roy99_ICRA, Alterovitz07_RSS, Kurniawati08_RSS, Guibas08_WAFR, Prentice09_IJRR, Platt10_RSS, vandenBerg11_ISRR, Patil11_RSS, Bry11_ICRA, Vitus11_ICRA}. %In addition to computing safe motion plans, these methods also compute control policies that compensate for uncertainty in the robot state.

A key step of many motion planning methods that consider uncertainty is to estimate the probability of collision of a motion plan. These planners generally characterize the uncertainty of the robot's position by estimating the a priori probability distributions of the robot state along a given plan. A common approach is to check for collisions between these distributions and obstacles to compute an upper bound for the collision probability, which is used as a metric to evaluate the safety of the plan \cite{vandenBerg11_IJRR, Patil11_RSS}.
%Estimating the probability of collision is an integral step in several motion planning methods that characterize uncertainty by estimating the a priori probability distributions of the robot state along a given plan.
This metric has also been used to penalize collisions with obstacles for computing optimal motion plans in an optimization based framework \cite{vandenBerg11_ISRR}. Vitus and Tomlin \cite{Vitus11_ICRA} use these distributions to compute a conservative probability bound using Boole's inequality for optimal motion planning using chance constrained optimization \cite{Blackmore09_GNCC}. Bry and Roy \cite{Bry11_ICRA} use an identical conservative estimate to compute edge costs for optimal motion planning in belief spaces. However, these methods incorrectly assume that the probabilities of collision are independent, which results in overly conservative plans or, in some cases, might result in failure to find a feasible plan even if one exists.

A na\"{i}ve Monte-Carlo sampling strategy could also be used to accurately estimate the probability of collision. Lambert et al.\ \cite{Lambert06_ICARV} and Du Toit and Burdick \cite{duToit11_TRO} propose methods to accelerate the performance of such methods. This approach is general enough to consider all sources of uncertainty, including motion and sensing uncertainty, and uncertainty in sensing the obstacles in the environment. However, this estimation process can be computationally expensive if performed repeatedly, like in \cite{vandenBerg11_ISRR, Vitus11_ICRA, Bry11_ICRA}. In contrast, we present a fast, analytical method for accurately estimating the conditional state distributions and the probability of collision of a plan with Gaussian models of motion and sensing uncertainty.

The use of truncated Gaussian distributions \cite{Book:Johnson94} has been previously explored by Simon \cite{Book:Simon06} in the context of optimal state estimation with state constraints, but this work does not consider motion uncertainty. Greytak \cite{Thesis:Greytak09} provides an analytical method to compute the probability of collision using truncated Gaussians but does not consider sensing uncertainty and only truncates the state distributions with respect to the obstacle closest to the robot state being considered. Toussaint \cite{Toussaint09_ICML} uses truncated Gaussians in an expectation-propagation framework for Bayesian inference under hard constraints and limits, but the truncation result is dependent on the order in which obstacles are processed, which leads to problems with convergence of the algorithm \cite{Toussaint09_NIPS}. In contrast, we propose a novel order-independent algorithm for truncating Gaussian distributions with respect to hard state constraints. 