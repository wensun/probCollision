% !TEX root =  ICRA2012-Patil.tex

%Explicitly considering uncertainty in robot motion and sensing during motion planning enables the computation of safer plans.

Motion planners that consider uncertainty have been developed for a variety of applications \cite{Alterovitz07_RSS, Kurniawati08_RSS, Guibas08_WAFR, Prentice09_IJRR, Platt10_RSS, vandenBerg11_IJRR, Patil11_RSS, Bry11_ICRA, Vitus11_ICRA}.
%
A key step of many motion planning methods that consider uncertainty is to estimate the probability of collision of a motion plan. Monte Carlo sampling strategies have been used to accurately estimate this probability \cite{Lambert06_ICARV, duToit11_TRO}. Other methods characterize the uncertainty by estimating the \emph{a priori} probability distributions of the robot state along a given plan. One approach is to check for collisions between these distributions and obstacles is to compute an upper bound for the collision probability for use as a metric to evaluate plan safety \cite{vandenBerg11_IJRR, Patil11_RSS}. Another approach is to use these distributions to compute a conservative probability bound using Boole's inequality \cite{Vitus11_ICRA, Bry11_ICRA}. However, these methods incorrectly assume that the probabilities of collision are independent, which results in overly conservative plans.%, or in some cases, might result in failure to find a feasible plan even if one exists.

The use of truncated Gaussian distributions \cite{Book:Johnson94} has been previously explored in the context of optimal state estimation with state constraints \cite{Book:Simon06}, but this work does not consider motion uncertainty. Greytak \cite{Thesis:Greytak09} provides an analytical method to compute the probability of collision using truncated Gaussians but does not consider sensing uncertainty. %and only truncates the state distributions with respect to the obstacle closest to the robot state being considered.
Toussaint \cite{Toussaint09_ICML} uses truncated Gaussians in an expectation-propagation framework for Bayesian inference, but the truncation result is dependent on the order in which constraints are processed, which leads to problems with convergence of the algorithm \cite{Toussaint09_NIPS}. In contrast, we propose a novel order-independent algorithm for truncating Gaussian distributions with respect to hard state constraints. 